\documentclass[a4paper, 12pt]{report}

%% -------- PREAMBLE -------- %%
\usepackage{cite}
\usepackage[spanish]{babel}
\usepackage{babel} 
\usepackage{apalike}
\usepackage[utf8]{inputenc}
\usepackage[T1]{fontenc}
\usepackage{microtype}
\usepackage[T1]{fontenc}
\usepackage{setspace} \doublespacing
\usepackage{fancyhdr, fancybox}
\usepackage{times}
\usepackage{graphicx}
\usepackage{blindtext}
\usepackage{multicol}
\usepackage{float}
\usepackage{titlesec}

\titleformat{\chapter}[display] {\normalfont\Huge\bfseries}{\chaptertitlename\ \thechapter}{0pt}{\Huge}
\titlespacing*{\chapter}{0pt}{0pt}{0pt}
\pagestyle{fancy}


\fancypagestyle{plain}{
    \fancyhf{} % clear all header and footer fields
    \renewcommand{\headrulewidth}{0pt}
    \renewcommand{\footrulewidth}{0pt}
    \fancyhead[R]{\large\thepage}
    \fancyheadoffset[R]{10pt}
}


\setlength{\headheight}{18.1pt}

\usepackage{listings}
\lstset{frame=tb,
  breaklines=true,
  showstringspaces=false,
  columns=flexible,
  numbers=none,
  commentstyle=\color{dkgreen},
  stringstyle=\color{mauve},
  tabsize=3
}
\usepackage[a4paper,
  total={6.27in, 9.69in}
  ]{geometry}
\usepackage{slashed}
\usepackage{changepage}
\usepackage{bbm}
\usepackage{setspace}  
\usepackage{float}
\usepackage{amsmath,amsthm,amssymb,amsfonts}
\usepackage[normalem]{ulem}
\usepackage{enumerate}
\usepackage{hyperref}
\usepackage[mathscr]{euscript}
\rfoot{\thepage}
\hypersetup{breaklinks=true}
\newcommand{\fsl}[1]{{\ooalign{\(#1\)\cr\hidewidth\(/\)\hidewidth\cr}}}

\newtheorem{defn}{Definición}[section]
\newtheorem{thm}{Teorema}
\newtheorem{lem}{Lema}[section]
\newtheorem{prop}{Proposición}[section]
\newtheorem{cor}[thm]{Corolario}
\newtheorem{ex}{Ejemplo}
\newtheorem{ob}{Observación}
\newcommand{\QED}{\hfill {\qed}}

%% -------- DOCUMENT -------- %%

\begin{document}
\pagestyle{empty}
\begin{center}
\Large\textbf{UNIVERSIDAD SAN FRANCISCO DE QUITO USFQ}\\
\vspace{1cm}
\large\textbf{Colegio de Ciencias e Ingenierías}

\vspace{4cm}
\begin{Large}
  \textbf{Estudio de uso de testores en la creación de redes de clasificación de imágenes para el modelado 3D de su rostro}
\end{Large}

\vspace{3cm}
\Large\textbf{Roberto Andrés Alvarado Moreira}\\
\vspace{0.5cm}
\Large\textbf{Matemáticas}\\

\vspace{1.5cm}
\large
Trabajo de fin de carrera presentado como requisito \\
para la obtención del título de \\
Matemático

\vfill
\normalsize Quito, \today


\setcounter{page}{1}
\thispagestyle{empty}
\end{center}

\newpage

\thispagestyle{empty}
\begin{center}

\Large\textbf{UNIVERSIDAD SAN FRANCISCO DE QUITO USFQ}\\
\vspace{0.3cm}
\normalsize\textbf{Colegio de Ciencias e Ingenierías}

\vspace{1.5cm}
\normalsize\textbf{HOJA DE CALIFICACIÓN\\
 DE TRABAJO DE FIN DE CARRERA}

\vspace{1.5cm}
\normalsize\textbf{Estudio de uso de testores en la creación de una red de clasificación de rostros para el modelado 3D}

\vspace{1cm}
\large\textbf{Roberto Andrés Alvarado Moreira}
\end{center}

\vspace{4cm}
\begin{minipage}{0.5\linewidth}   
  \normalsize Julio César Ibarra Fiallo Ph.D. 
  \vspace{1.4cm}\\
  \\
  \normalsize Antonio Di Teodoro, Ph.D.\\
\end{minipage}
\begin{minipage}{0.3\linewidth}
  \normalsize ............................................\\
  \vspace{1.2cm}\\
  \normalsize ............................................
\end{minipage}

\vfill 
\begin{center}
\normalsize Quito, \today
\end{center}

\newpage
\begin{center}
\begin{large}
\copyright \hspace{0.1cm} \textbf{DERECHOS DE AUTOR}
\end{large}
\end{center}

\normalsize Por medio del presente documento certifico que he leído todas las Políticas y Manuales de la Universidad San Francisco de Quito USFQ, incluyendo la Política de Propiedad Intelectual USFQ, y estoy de acuerdo con su contenido, por lo que los derechos de propiedad intelectual del presente trabajo quedan sujetos a lo dispuesto en esas Políticas. 

\normalsize Asimismo, autorizo a la USFQ para que realice la digitalización y publicación de este trabajo en el repositorio virtual, de conformidad a lo dispuesto en la Ley Orgánica de Educación Superior del Ecuador.



\vspace{5cm}
\noindent
\begin{minipage}{0.3\linewidth}   
  \small Nombres y apellidos:
  \vspace{0.7cm}

  \small Código:
  \vspace{0.7cm}

  \small C.I.:
  \vspace{0.7cm}

  \small Fecha:
  \vspace{0.7cm}

\end{minipage}
\begin{minipage}{0.7\linewidth}
  \small Roberto Andrés Alvarado Moreira
  \vspace{0.7cm}\\
  \small 00206411
  \vspace{0.7cm}\\
  \small 1104718141
  \vspace{0.7cm}\\
  \small Quito, \today
  \vspace{0.7cm}

\end{minipage}


\newpage
\begin{center}
\begin{LARGE}
\textbf{ACLARACIÓN PARA PUBLICACIÓN}
\end{LARGE}
\end{center}


\noindent
\normalsize\textbf{Nota:} El presente trabajo, en su totalidad o cualquiera de sus partes, no debe ser considerado como una publicación, incluso a pesar de estar disponible sin restricciones a través de un repositorio institucional. Esta declaración se alinea con las prácticas y recomendaciones presentadas por el Committee on Publication Ethics COPE descritas por Barbour et al. (2017) Discussion document on best practice for issues around theses publishing, disponible en \url{http://bit.ly/COPETheses}

\setlength{\parskip}{10mm}
\begin{center}
\begin{LARGE}
\textbf{UNPUBLISHED DOCUMENT}
\end{LARGE}
\end{center}


\noindent
\normalsize\textbf{Note:} The following capstone project is available through Universidad San Francisco de Quito USFQ institutional repository. Nonetheless, this project – in whole or in part – should not be considered a publication. This statement follows the recommendations presented by the Committee on Publication Ethics COPE described by Barbour et al. (2017) Discussion document on best practice for issues around theses publishing available on \url{http://bit.ly/COPETheses}




\newpage
\begin{center}
  \large \textbf{RESUMEN}
\end{center}

\noindent
\textbf{\textit{Palabras clave}:} Redes Neuronales, Modelos de Clasificación,
Testores, Conjuntos Rugosos, Reductos, UMDA, Redes Generativas.
 

\newpage
\begin{center}
  \large \textbf{ABSTRACT}
\end{center}


\textbf{\textit{Keywords}:} First order model with delay, transfer function without delay, fractional calculus, sliding mode control, time delay approximation, fractional control law, Mittag-Leffler.

\newpage
\tableofcontents % indice de contenidos

\newpage
\listoffigures % indice de figuras
\newpage
\section*{Dedicatoria}
%\newpage
\begin{flushright}
  \Large A mis padres, mis hermanas, mi familia y mis amigos, \\
  que sin esperar nada me apoyaron
\end{flushright}

\newpage
\begin{center}
  \textbf{AGRADECIMIENTO}
\end{center}
Quiero expresar mi profundo agradecimiento a mi tutor de tesis,
Julio Ibarra, por su invaluable orientación, apoyo incondicional y
paciencia durante todo el proceso de investigación. Gracias a su experiencia y
conocimientos, que han logrado formarme de la mejor manera posible. 
A David Hervas, mi primer profesor de la universidad, que
con su manera tan especial de ver las matemáticas logró contagiar en mi el amor
por ellas.

Además quiero agradecer todas las experiencias que pude conseguir a lo
largo de mi vida, me han hecho crecer como persona y en ellas he conocido a
gente maravillosa, quiero agradecerles a:

A mi papá, a mi mamá y mis hermanas, a toda mi familia, a la USFQ y todos mis
profesores; a Gerson, Andrés V y Natanael, a Marco, Eduarda, Alejandra, Camila V,
Camila P, Mille, Meli y Andrés, a Pablo, José, Adrián y Ricardo, a Mateo y
Alejandro.

 

\clearpage
%\addcontentsline{toc}{chapter}{Lista de Figuras} % para que aparezca en el indice de contenidos

\chapter{Introduccion}\label{cap.introduccion}
Dentro del estudio de las redes neuronales el uso de testores se muestra como
uno de los nuevos métodos para la optimización y manejo de los modelos de
clasificación, se

\cite{testor}

\chapter{Definiciones y Métodos}\label{cap.testores}

\addcontentsline{toc}{chapter}{Bibliografía}


\bibliographystyle{apalike}
\bibliography{bibliography} 

\end{document}
